\subsection{Virtual Method Invocation}
\textit{Write code to show how virtual method invocation lets one implementation be swapped for another.}

In section \ref{s:inheritance} on page \pageref{s:inheritance} we looked into polymorphism and inheritance.

Figure \ref{fig:inheritance} on page \pageref{fig:inheritance} shows that the \texttt{Animal} class has an \texttt{abstract} method \texttt{say()} and the \texttt{Dog} and \texttt{Cat} classes implement it.

When we create a \texttt{Cat} or \texttt{Dog} object and assign it to an \texttt{Animal} reference:
\begin{lstlisting}[language=Java]
Animal pluto = new Dog("Pluto", Animal.Sociability.VERY_SOCIAL, 3);
Animal sheba = new Cat("Sheba", 2);
\end{lstlisting}

We can now make either the \texttt{Cat} or Dog say something:
\begin{lstlisting}[language=Java]
pluto.say();
sheba.say();
\end{lstlisting}
And the result will be:
\begin{lstlisting}
Bark (wag) (wag) (wag)
Meouw (cough) (cough)
\end{lstlisting}

Although we call the Animal.say() method, it really executes the version implemented by the child classes. 