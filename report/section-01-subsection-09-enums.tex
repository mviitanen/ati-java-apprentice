\subsection{Enums}
\textit{Write your own enum type.  Describe when you would use it.}

An \texttt{enum} is a special object that looks like a constant. They are useful when we need a list of constants that are not numeric. For example, types of things are well represented as an \texttt{enum}, such as representing a gender of a person. Instead of having numbers (0=unknown, 1=male, 2=female) or \texttt{String}s ("unknown", "male", or "female") we should use an \texttt{enum}. If we use numbers or \texttt{String}s, for example, we don't have control over the possible values that the user might assign to the value. What would we do it someone assigns 99 to gender (\texttt{int})? If we chose to do that, we would need extra logic to verify the value. Also, the user doesn't know which values are valid before trying it, or reading the documentation.

\subsubsection{A Simple Enum}
A simple \texttt{enum} is basically a list of names:
\begin{lstlisting}[language=Java]
public enum NaughtyOrNice {
  REALLY_NICE,
  NICE,
  NOT_SO_NICE,
  AVERAGE_NICE,
  NOT_REALLY_NAUGHTY,
  NAUGHTY,
  REALLY_NAUGHTY
}
\end{lstlisting}

Each \texttt{enum} value is assigned an ordinal (\texttt{int}). This value can be retrieved, but it cannot be set. We shouldn't rely on the ordinal value, as it might change when we add or remove \texttt{enum} values.

\begin{lstlisting}[language=Java]
NaughtyOrNice status = NaughtyOrNice.AVERAGE_NICE;
System.out.println(status + ":{\n"
 + " name:'" + status.name() + "',\n"
 + " ordinal:'" + status.ordinal() + "'\n" +
 "}");

if (peterStatus.equals(NaughtyOrNice.REALLY_NAUGHTY)) {
 System.out.println("He has no hope");
} else {
 System.out.println("He has some hope");
}
\end{lstlisting}

When run, the above code will produce:
\begin{lstlisting}
AVERAGE_NICE:{
 name:'AVERAGE_NICE',
 ordinal:'3'
}
He has some hope
\end{lstlisting}

\subsubsection{A Custom Enum}
There are cases when we want the \texttt{enum} to have values associated to it. For example, we might want to associate strings with an \texttt{enum} for parsing or displaying.

A custom \texttt{enum} looks like a class, except that it its type is \texttt{enum} not \texttt{class}.

\begin{lstlisting}[language=Java]
public enum ErrorType {
 //...
 
 public int getErrorCode() {
  return errorCode;
 }

 public String getErrorMessage() {
  return errorMessage;
 }

 @Override
 public String toString() {
  return this.name() + ": {\n"
   + " ordinal:'" + ordinal() + "',\n"
   + " msg:'" + errorMessage + "',\n"
   + " code:'" + errorCode + "'\n" +
   "}";
 }
}
\end{lstlisting}

Declaring the possible values is done in a strange way too, inside the \texttt{enum} itself:
\begin{lstlisting}[language=Java]
E404(404, "Not Found"),
E500(500, "Really Bad Error"),
UNKNOWN(0, "Unknown");
\end{lstlisting}
In the above example, we declare the name of the \texttt{enum} value (E404, for example), then pass in any parameters to the constructor. In this case we assign a numeric error code (404) and a message string "Not Found".

See the source for the custom \texttt{enum} in Appendix I (ErrorType.java) on page \pageref{App:AppendixIEnum}.

Sometimes it is useful to map a literal value with an \texttt{enum}. For example, if we get an error from an API in form of a number, we might want to map it into an \texttt{enum}. Unfortunately this is not provided. The following code shows how to do it:
\begin{lstlisting}[language=Java]
public static ErrorType fromValue(int errorCode) {
 ErrorType[] values = ErrorType.values();
 for (ErrorType value : values) {
  if (value.getErrorCode() == errorCode) {
   return value;
  }
 }
 return UNKNOWN;
}
\end{lstlisting}
The code iterates through the values of the \texttt{enum} and picks the one that matches the given input. If none is found, then it defaults to \texttt{UNKNOWN}.

To convert a numeric value into an \texttt{enum} can then be done:
\begin{lstlisting}[language=Java]
ErrorType error500 = ErrorType.fromValue(500);
System.out.println(error500);
\end{lstlisting}

The output will be:
\begin{lstlisting}
E500: {
 ordinal:'1',
 msg:'Really Bad Error',
 code:'500'
}
\end{lstlisting}

\subsubsection{Switching on an Enum}
We can use \texttt{enums} in a switch statement:
\begin{lstlisting}[language=Java]
String msg;
switch (myError) {
 case E404:
  msg = "Where did it go?";
  break;
 case E500:
  msg = "Let's just give up";
  break;
 default:
  msg = "Who knows?";
  break;
}
\end{lstlisting}
However, the syntax is a bit restricted. For example You cannot use a fully qualified \texttt{enum}
\begin{lstlisting}[language=Java]
case ApplicationError.E404:
\end{lstlisting}
Although this is valid in an if statement, for example, this causes the compiler to complain that the case label is not a constant. The proper way is:
\begin{lstlisting}[language=Java]
case E404:
\end{lstlisting}
Also wrapping the cases in parenteses causes an error:
\begin{lstlisting}[language=Java]
case (E404):
\end{lstlisting}
generates an error "Constant expression required". Parentheses around an \texttt{int} case label, for example is valid syntax. Not when it comes to enums.

See full listing in Appendix I (EnumExample.java) on page \pageref{App:AppendixIExample}