\subsection{StringBuffer vs StringBuilder}
\textit{Compare and contrast StringBuffer and StringBuilder and when to use each}

StringBuffer and StringBuilder have identical APIs. The main difference between them is that a StringBuffer is thread-safe and StringBuilder is not.

StringBuffer is the older of the two and was created with synchronization so it could be used by several threads. In practice, it is hard to find a real use case for that. In theory, StringBuilder, that is not thread-safe, is faster because it doesn't have to worry about concurrency.

\subsubsection{Concurrent Writes}
In order to test the concurrency  we can create two threads that write into a \texttt{StringBuffer} and a \texttt{StringBuilder} 10 times. See Appendix D on page \pageref{App:AppendixD} for full source code.

Each thread writes into the buffer, then into the builder, and then sleeps for 10ms. In the end it prints out the contents of the StringBuilder and the StringBuffer. In listing \ref{lst:sb-concurrency} we can see the output of the test. There are two columns, one for a StringBuffer and another for the StringBuilder.  
\begin{lstlisting}[caption=StringBuffer and StringBuilder Concurrency Test, label=lst:sb-concurrency]
Buffer	          Builder
thread 0 -> 0     thread 1 -> 0
thread 1 -> 0     thread 0 -> 0
thread 0 -> 1     thread 0 -> 1
thread 1 -> 1     thread 1 -> 1
thread 0 -> 2     ????thread 0 -> 2
thread 1 -> 2     thread 1 -> 3
thread 1 -> 3     thread 0 -> 3
thread 0 -> 3     thread 1 -> 4
thread 1 -> 4     thread 0 -> 4
thread 0 -> 4     thread 1 -> 5
thread 1 -> 5     thread 0 -> 5
thread 0 -> 5     thread 1 -> 6
thread 1 -> 6     thread 0 -> 6
thread 0 -> 6     thread 1 -> 7
thread 1 -> 7     thread 0 -> 7
thread 0 -> 7     thread 1 -> 8
thread 1 -> 8     thread 0 -> 8
thread 0 -> 8     thread 1 -> 9
thread 1 -> 9     thread 0 -> 9
thread 0 -> 9     
\end{lstlisting}
\vspace{2em}

The contents of the StringBuilder and StringBuffer are different. Not only is the order different but the StringBuilder seems to be missing a line. Thread 1, iteration 2 is missing in the StringBuilder output. Instead there are non-printing characters (shown as '?'), on line 6 of listing \ref{lst:sb-concurrency}.

Looking at the bytes in the StringBuilder shows the issue.

Each line consists of four characters; the thread name (really a number from 0 to 9), an arrow, and the counter value (from 0 to 9), and a new line character (hex 0a), like this:
\begin{lstlisting}
7468 7265 6164 2030 202d 3e20 370a thread 0 -> 7.
\end{lstlisting}

There are several issues in the content of the StringBuilder, though. The very first writes collided. Thread 0 and 1 output is mangled:
\begin{lstlisting}
7468 7265 6164 2031 202d 3e20 7468 7265  thread 1 -> thre
6164 2030 202d 3e20 300a 300a            ad 0 -> 0.0.
\end{lstlisting}

Also the StringBuilder contains exactly four bytes of 0's, though, starting in position 71 (hex 47), middle of line 5 in listing \ref{lst:sb-collision}.
\begin{lstlisting}[caption=StringBuilder Collision, label=lst:sb-collision]
0000: 7468 7265 6164 2031 202d 3e20 7468 7265
0010: 6164 2030 202d 3e20 300a 300a 7468 7265
0020: 6164 2031 202d 3e20 310a 7468 7265 6164
0030: 2030 202d 3e20 310a 7468 7265 6164 2074
0040: 202d 3e20 6420 0000 0000 3120 2d3e 2032
0050: 320a 7468 7265 6164 2031 202d 3e20 7468
...
\end{lstlisting}
\vspace{2em}

It looks like the two threads collided. Thread 1 reserved 4 bytes in the StringBuilder, but thread 2 wrote its output. This clearly shows how StringBuilder is not thread safe.

\subsubsection{Comparison Test}
We can compare appending to a StringBuffer and a StringBuilder to see which one is faster, using different lengths of strings and different number of additions. The test program appends strings of 1, 5, 10, 25, and 50 characters long into a StringBuffer or a StringBuilder. It appends from 1,000,000 to 10,000,000 strings per run, in increments of 1,000,000 strings. See appendix E on page \pageref{App:AppendixE}.

In the graph below, we can see that as for both the StringBuffer and the StringBuilder, when the size of the string to append increased, the time increased also. Even greater was the time increase as the number of additions grew. In those aspects both the StringBuffer and the StringBuilder behaved similarly.

Suprisingly, appending into a StringBuilder was not always faster than into a StringBuffer. In the graph, the blue plane is the data of appending into a StringBuffer. The red dots show when appending into a StringBuilder was slower. Although the difference is not great, there is no clear pattern of why the StringBuilder was slower sometimes.

\pgfplotstableread{buffer.txt}{\plotdata}
\begin{figure}[H]\centering
\begin{framed}
\begin{tikzpicture}
  \begin{axis}[view/h=40, grid=both, xlabel={String Size}, ylabel={Additions ($10^7$)}, zlabel={Time (ms)}, ylabel shift=-15pt, xlabel shift=-10pt, zlabel shift=-6pt,ytick scale label code/.code={}]
\addplot3[surf, opacity=0.8, colormap=
    {blackblue}{color=(blue) color=(blue)}] file {buffer.txt};    
    \addplot3[red!80,mark=*,
mark options={fill=red!80!red},
only marks,mark size=1pt] file {builder2.txt};    
  \end{axis}
\end{tikzpicture}
\end{framed}
\label{fig:stringbuildervsStringBuffer}
\end{figure}

Based on the test program, there is no real difference between the performance of a StringBuffer and a StringBuilder.
