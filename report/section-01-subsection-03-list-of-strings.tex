\subsection{List of Strings}
\textit{Write an application to find out how many total characters can be held in a list of strings before you run out of memory.}

The number of characters we can store in a list depends on many factors; the length of the strings, whether they are String literals or different objects, or the size of the heap. 

\subsubsection{Heap Size}

We can query the size of the heap, and the amount of free memory:
\begin{lstlisting}[language=Java]
Runtime rt = Runtime.getRuntime();
long totMem = rt.totalMemory();
long freeMem = rt.freeMemory();
\end{lstlisting}

The default size of the heap is at most 1GB. If one quarter of the physical memory is smaller than 1GB, then that value is used\cite{gcergo}. We can query the amount of physical memory:
\begin{lstlisting}[language=Java]
OperatingSystemMXBean mxbean = (OperatingSystemMXBean) ManagementFactory.getOperatingSystemMXBean();
System.out.println(mxbean.getTotalPhysicalMemorySize()/1024/1024);
// 16384
\end{lstlisting}
See Appendix C for the complete source of \texttt{HeapSize\hyp{}Example.java}.
We get the following values on a MacBook Pro with 16GB of RAM:

\begin{table}[!htb]
\centering
\begin{tabulary}{\columnwidth}{ | l | r | r |}
\hline
\textbf{Type of Memory} &  \textbf{Bytes}& \textbf{MB} \\ \hline 
Total Available & 257,425,408 & 245 \\ \hline
Free & 252,056,464 & 240 \\ \hline
Used & 5,368,944 & 5 \\ \hline
Physical & 17,179,869,184 & 16,384 \\ \hline
\end{tabulary}
\caption{Querying the Memory}\label{tab:memory}
\end{table}

One fourth of the physical memory is 4GB. In this case, because one quarter of the physical memory is larger than 1GB, the default heap is the smaller value, or 1GB.

The maximum heap size on a MacBookPro:
\begin{lstlisting}[language=bash]
$ java -XX:+PrintFlagsFinal -version 2>&1 | grep MaxHeapSize
uintx MaxHeapSize := 4294967296 {product}
\end{lstlisting}
So, approximately 4.29GB. 

\subsubsection{Size of a String}

Java doesn't have a sizeof() functionality like C has, but we can estimate the size of objects in the heap. For a Java String, two bytes are used for each character. There is a bit of overhead (which varies depending on whether the OS is a 32 or 64-bit architecture, for example), and rounding to the next usable memory boundary.

So, a String of 200 characters is approximately 440 bytes in memory (200*2bytes + 38 overhead, and rounded to the closest 8-bit boundary). And empty String (with just overhead) is 40 characters.

We can allocate \texttt{String}s, query the amount of used heap before and after, and find out how much memory (approximately) the \texttt{String}s took. If we request garbage collection (although not guaranteed), and allocate Strings of 200 characters until we run out of memory (7,480,318 strings in my example program), the result of the heap is:

\begin{table}[!htb]
\centering
\begin{tabulary}{\columnwidth}{ |>{\raggedright\arraybackslash} p{0.5\columnwidth} | >{\raggedright\arraybackslash}r |}
\hline
Used memory before  & 356,352 bytes \\ \hline 
Used memory after & 3,321,618,024 bytes \\ \hline 
Delta & 3,321,261,672 bytes\\ \hline 
Number of strings  & 7,480,318 \\ \hline
Heap per a string & 444 bytes \\ \hline 
\end{tabulary}
\caption{String Average Size}\label{tab:string-size}
\end{table}

We can see that in this example, the average size of a \texttt{String} is 444 bytes.

\subsubsection{Heap and Stack Exceptions}
There are several types of exceptions related to running out of heap or stack.

A \texttt{java.lang.StackOverflowError} is thrown when we run out of stack space. This can occur when we make too many method calls (add too many frames) recursively. In these String List examples, we will not likely run into this problem because we only make a couple method calls, thus not changing the scope a lot.

A \texttt{java.lang.OutOfMemoryError} is thrown when we run out of heap, or when the garbage collection cannot efficiently release more heap memory. It is guaranteed to be thrown before we actually run out of heap. The error message can give us more information why the error was thrown. "\texttt{Java heap space}" message means that we truly ran out of heap. "\texttt{GC overhead limit exceeded}" message means that the garbage collection is making poor progress. We get that error if "the Java process is spending more than approximately 98\% of its time doing garbage collection and if it is recovering less than 2\% of the heap and has been doing so far the last 5 (compile time constant) consecutive garbage collections" \cite{gosling}.

If we create a lot of small Strings, we could run into the issue of garbage collection being inefficient before actually running out of heap.

\subsubsection{Test Program}
We can test how many \texttt{String}s or characters we can create before running out of heap space. We can test several cases. For example, we can use small Strings (5 characters), medium Strings (100 characters), large Strings (1000 characters), and a random mix of lengths (1-1000 characters). We can also test with internal Strings (literal) and dynamic Strings (created with the new keyword). So, the test cases are:
\begin{itemize}
\item Small  literal Strings
\item Small dynamic Strings
\item Medium literal Strings
\item Medium dynamic Strings
\item Large literal Strings
\item Large dynamic Strings
\item Random literal Strings
\item Random dynamic Strings
\end{itemize}

See the source code for the test (\texttt{ListOfStrings\hyp{}Example.java}) in Appendix B, page \pageref{App:AppendixB}.

\subsubsection{The Results}
The first test involved using internal (literal) Strings of different sizes. The actual String data is exactly the same copy. In practice, this is a useless example, but it is nevertheless interesting. 

We have a reference to the List in the stack. The List has references to the String items in the heap. The List object grows to fill the entire heap until we get an OutOfMemoryError. We don't run out of the stack so we never get a StackOverFlowException.

The results of adding literal strings to a list:
\begin{table}[H]
\centering
\begin{tabulary}{\columnwidth}{ | r | r | r |}
\hline
\textbf{Length} & \textbf{Strings}& \textbf{Chars}\\ \hline 
5 & 287,568,537 & 1,437,842,685 \\ \hline
200 & 287,568,537 & 57,513,707,400 \\ \hline
1000 & 287,568,537 & 287,568,537,000 \\ \hline
\end{tabulary}
\caption{List of Literal Strings}\label{tab:listOfLiteralStrings}
\end{table}

We can see that because there is only one exact copy of the actual String data and the List just contains references to the exact same String, the heap runs out of memory with exactly the same number of items in the List, regardless of the String size. The number of items in the List is constant. Because of the size of the String doesn't matter when using a literal String, we can "fit" more characters in the List if the Strings are bigger.

The results of adding dynamic strings to a list:
\begin{table}[!htb]
\centering
\begin{tabulary}{\columnwidth}{ | r | r | r |}
\hline
\textbf{Length} &  \textbf{Strings}& \textbf{Chars} \\ \hline 
5 & 55,591,797 & 277,958,985 \\ \hline
200 & 7,480,318 & 1,496,063,600 \\ \hline
1000 & 1,631,552 & 1,631,552,000 \\ \hline
1-1000 & 3,194,418 & 1,594,871,852 \\ \hline
\end{tabulary}
\caption{List of Dynamic Strings}\label{tab:listOfDynamicStrings}
\end{table}

With dynamic Strings, we can see that as the size of the String gets bigger, the number of items in the List gets smaller. The List full of small Strings holds the least number because most of the heap is used for String overhead. The List with random length Strings is logically between a list of 200-character-String List and 1000-character-String List.

\subsubsection{Overhead Ratio}
There is a point where the overhead becomes less and less relevant. As the String size grows, the number of characters, or Strings, becomes practically linear. 
\pgfplotsset{width=0.75\columnwidth,compat=1.3}\\

\begin{figure}[H]\centering
\begin{framed}
\begin{tikzpicture}
\pgfplotsset{
    scale only axis,
%    	xmajorgrids,
    xmin=0, xmax=1000,
    legend style={at={(1,0.5)},anchor=east}
}
\begin{axis}[
    axis y line*=left,
    axis x line=none,
    ylabel={\# of Strings},
    ymin=0, ymax=70000000
]
% # of strings
\addplot[color=red, only marks, smooth,dashed]
    coordinates {
(5,   69429204)
(10,  56803662)
(20,  50864895)
(30,  43559683)
(40,  37869108)
(50,  34779731)
(60,  31345545)
(70,  28185086)
(80,  26173543)
(90,  24571971)
(100, 22794732)
(150, 16830715)
(200, 13527266)
(250, 11220477)
(300,  9636571)
(350,  8407166)
(400,  7480318)
(450,  6729742)
(500,  6095017)
(600,  5154652)
(700,  4474650)
(800,  3945494)
(900,  3521442)
(1000, 3184637)

    };
    \label{strings_plot}
\addlegendentry{\# of Strings}

\end{axis}

\begin{axis}[
    axis y line*=right, 
    xlabel={String Length},
    ylabel={\# of Characters},
%    	ymajorgrids,
    ymin=0, ymax=1600000000
	]
\addlegendimage{/pgfplots/refstyle=strings_plot}\addlegendentry{\# of Strings}
% # of characters
\addplot[color=blue, only marks, smooth]
    coordinates {
(5,   138860440)
(10,  255609450)
(20,  483226293)
(30,  631621316)
(40,  738473565)
(50,  852200636)
(60,  924632662)
(70,  972560340)
(80,  1033822001)
(90,  1093504365)
(100, 1128493640)
(150, 1253741942)
(200, 1346137653)
(250, 1396506412)
(300, 1440740739)
(350, 1467137115)
(400, 1491908107)
(450, 1510647853)
(500, 1520613428)
(600, 1544464935)
(700, 1564055520)
(800, 1575428971)
(900, 1582875340)
(1000,1591165537)
    };
 \label{characters_plot}
\addlegendentry{\# of Characters}
\end{axis}
\end{tikzpicture}
\end{framed}
\caption{Number of Characters in a List by String Length}
\label{fig:listofstrings}
\end{figure}
\pgfplotsset{width=\columnwidth,compat=1.3}

The test program threw a heap exception when the heap was at 3.09GB. With that heap size, the number of characters approaches 1,600,000,000, which seems close to the maximum number of characters we can store in a List of Strings (with this heap size).

The number of Strings in the List approaches 3,000,000.

When the String length is greater than about 200, the overhead seems to get less and less in the way.