\subsection{Casting}
\textit{Write code that uses the instanceof operator and show how casting works.}

In section \ref{s:inheritance} on page \pageref{s:inheritance} we created two \texttt{Animal} references, one pointing to a \texttt{Dog} object (\texttt{pluto}) and another pointing to a \texttt{Cat} object (\texttt{sheba}).

In figure \ref{fig:inheritance} on page \pageref{fig:inheritance} we can see that the \texttt{Dog} class also has a \texttt{wagCount} field and the \texttt{Cat} has a \texttt{furballCount} field. They both provide getters and setters for those attributes. The \texttt{Animal} class doesn't have either \texttt{wagCount} or \texttt{furballCount}.

If we assign the \texttt{Cat} and \texttt{Dog} objects to an \texttt{Animal} reference, we cannot simply access those fields in the \texttt{Cat} and \texttt{Dog} classes. 

\begin{lstlisting}[language=Java]
Animal animal = new Dog("Pluto", Animal.Sociability.VERY_SOCIAL, 3, tail);
// won't compile
//pluto.getWagCount();
}
\end{lstlisting}

We need to cast down first. This means that we explicitly convert the \texttt{Animal} reference to either a \texttt{Dog} or \texttt{Cat} reference. That way we get access to the specific methods in the child classes.

Casting is done by prefixing the object reference with the new type, in parentheses: 

\begin{lstlisting}[language=Java]
Dog pluto = (Dog)animal;
}
\end{lstlisting}

We could now call the \texttt{Dog} methods. We also have access to the \texttt{Animal} methods because \texttt{Dog} inherits from \texttt{Animal}. It still "is-an" \texttt{Animal}. Just the reference changed.

What if animal is a \texttt{Cat}? We can query the type of the reference with the \texttt{instanceof} operator. It will return a \texttt{boolean} if an object is an instance of a class.

\begin{lstlisting}[language=Java]
System.out.println(pluto.say());
if(pluto instanceof Dog) {
 ((Dog) pluto).setWagCount(10);
 System.out.println(pluto.say());
}
\end{lstlisting}

If we try to cast an \texttt{Animal} reference, that points to a \texttt{Dog}, into a \texttt{Cat}, we will get a \texttt{ClassCastException}:
\begin{lstlisting}
Exception in thread "main" java.lang.ClassCastException: org.familysearch.viitanenm.Dog cannot be cast to org.familysearch.viitanenm.Cat
...
\end{lstlisting}

\subsubsection{A Generic Way}

The problem with the \texttt{instanceof} operator is that we have to know beforehand what the child type might be. Sometimes we want to make that decision in runtime. Let's consider the following:

\begin{lstlisting}[language=Java]
if(pluto instanceof sparky.getClass()){
  // do something
}
\end{lstlisting}

This produces a compilation error. 

There is a method \texttt{isAssignableFrom()} that can be used if we don't know the class type until at runtime.
\begin{lstlisting}[language=Java]
if(pluto.getClass().isAssignableFrom(sparky.getClass())){
 // do something
}
\end{lstlisting}

Using \texttt{isAssignableFrom()} compiles fine. See Appendix M on page \pageref{App:AppendixM} for the source code of the \texttt{CastingExample.java} class.