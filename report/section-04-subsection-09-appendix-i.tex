\subsection*{Appendix I -- Enums} \label{App:AppendixI}
\addcontentsline{toc}{subsection}{Appendix I} % Adds this section to the table of contents
\qrcode[height=.5in]{http://google.com}
\subsubsection{Using Enums}
\begin{lstlisting}[language=Java]
package org.familysearch.viitanenm;

public class EnumExample {

  public enum NaughtyOrNice {
    REALLY_NICE,
    NICE,
    NOT_SO_NICE,
    AVERAGE_NICE,
    NOT_REALLY_NAUGHTY,
    NAUGHTY,
    REALLY_NAUGHTY
  }

  public static void main(String[] args) {
    new EnumExample().doIt();
  }

  private void doIt() {
    NaughtyOrNice peterStatus = NaughtyOrNice.AVERAGE_NICE;
    System.out.println(peterStatus + " " + " {name:'" + peterStatus.name()+ "', ordinal:'" + peterStatus.ordinal() +"'}");

    if(peterStatus.equals(NaughtyOrNice.REALLY_NAUGHTY)){
      System.out.println("He has no hope");
    }
    else {
      System.out.println("He has some hope");
    }

    // Using a custom enum
    ApplicationError error404 = ApplicationError.E404;
    ApplicationError error500 = ApplicationError.fromValue(500);

    // Using an enum in a switch
    System.out.println();
    String msg;
    switch (error404) {
      case E404:
        msg = " -- Where did it go?";
        break;
      case E500:
        msg = " -- Let's just give up";
        break;
      default:
        msg = " -- Who knows?";
        break;
    }

    System.out.println(error404.toString() + msg);
    System.out.println(error500.toString() + " {name:'" + error500.name()+ "', ordinal:'" + error500.ordinal() +"'}");
  }
}

\end{lstlisting}

\subsubsection{A Specialized Enum}\label{App:AppendixISpecialEnum}
\begin{lstlisting}[language=Java]
package org.familysearch.viitanenm;

/**
 * Created by viitanenm on 12/9/16.
 */
public enum ApplicationError {
  E404(404, "Not Found"),
  E500(500, "Really Bad Error"),
  UNKNOWN(0, "Unknown");

  private int errorCode;
  private String errorMessage;

  ApplicationError(int errorCode, String errorMessage) {
    this.errorCode = errorCode;
    this.errorMessage = errorMessage;
  }

  public static ApplicationError fromValue(int errorCode) {
    ApplicationError[] values = ApplicationError.values();
    for (ApplicationError value : values) {
      if (value.getErrorCode() == errorCode) {
        return value;
      }
    }
    return null;
  }

  public int getErrorCode() {
    return errorCode;
  }

  public String getErrorMessage() {
    return errorMessage;
  }

  /**
   * Returns the name of this enum constant, as contained in the
   * declaration.  This method may be overridden, though it typically
   * isn't necessary or desirable.  An enum type should override this
   * method when a more "programmer-friendly" string form exists.
   *
   * @return the name of this enum constant
   */
  @Override
  public String toString() {
    return this.getErrorMessage() + "(" + this.getErrorCode() + ")";
  }
}
\end{lstlisting}